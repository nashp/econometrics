\section{Introduction}
This paper outlines methods to forecast commodity future returns using the commodity convenience yield or adjusted basis, the difference between the spot price and given future price of a commodity adjusted by the interest rate. We find some out of sample predictive power in this signal across some commodities and currencies across medium term horizons but not in the short term generally. We investigate the basis across a range of liquid futures markets and we investigate the efficacy of two models, ordinary least squares to predict future currency returns and ARIMAX to determine forecast future exchange rates. We verify the results in \cite{mainref} and expand on these by increasing the universe of commodities and currencies under investigation. We also include an ARIMAX model as an additional benchmark.

There is extensive literature on forecasting exchange rates and these can typically be broken down into distinct categories, those that use a particular model, Purchasing Power Parity for example as in \cite{ecbnapkin} or those that attempt to use economic variables to determine exchange rate returns as in \cite{meese}. However, it is widely noted that it is difficult to beat a random walk process for determining exchange rates using either of the methodologies see \cite{ecbnapkin}, \cite{engel} or indeed as first shown by \cite{meese}. We will focus specifically on the the convenience yield, avoiding analysis of the impact of commodity future prices \& returns themselves as this has been consistently proven as a poor forecast of exchange rates across the literature (\cite{oilpricrogoff}, \cite{engel}, \cite{mainref}). In narrowing our scope to commodity currencies we hope to validate and extend the findings shown in \cite{mainref}. 

\cite{engel} argue that there exists a linear relationship between the exchange rate and economic fundamentals or shocks. They show that the  exchange rate is the expected
present discounted value of this combination. They note if these fundamentals are of I(1) or exhibit a unit root or close to a unit root then the process exhibits the properties of a random walk. Whilst not a typical economic variable we use the commodity basis or convenience yield\footnote{The literature widely refers to both the commodity basis and the convenience yield, however there is a subtle difference. The basis is merely the difference between the spot price and some future price. The convenience yield is not directly observable but can be estimated by adjusting the basis for interest rates and storage costs and indeed indicates the benefit or value of holding physical inventory}. We posit that because the convenience yield accurately reflect the contemporaneous supply and demand conditions for a given commodity \cite{imfsupplyshock}, \cite{gorton2007}, where a higher convenience yield benefits holders or producers of the physical commodity, in this instance, commodity producing countries. Macro-economic factors also affect the convenience yield as shown by \cite{cyielddetermin}, particularly non-grain commodities, embodying further information that may determine future exchange rate returns. We also show the convenience yield is I(0) across all commodities studied

\textbf{Refer to results a bit more}

The paper is structured as follows. \Cref{sec:theory} discusses the theory of some of the aspects used in the construction of the model and \autoref{sec:data} discusses the data. We discuss the model in \autoref{sec:model} and subsequent results in \autoref{sec:results}. The final \autoref{sec:conclusion} is reserved for further commentary and a conclusion.




