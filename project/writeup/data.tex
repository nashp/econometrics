\section{Data}
\label{sec:data}
\subsection{Foreign Exchange \& Country Selection}
We focus on a number of countries and currencies described in the literature as ``Commodity Currencies'' by \cite{cocurrencies} amongst others. These are countries where a significant portion of their terms of trade are made up of commodity exports, particularly raw materials. The impact of the commodity price fluctuations are discussed in \cite{coboomexport} and we select a sample of countries from this paper, based on the liquidity as per the Bank of International Settlements (BIS) \cite{bisturnover} of their currency markets and a floating exchange rate to the US Dollar (USD). In addition to the sample of emerging economies we include the non-emerging economies of Australia, Norway and Canada. We use the BIS USD exchange rate provided by\textit{dbnomics} \cite{dbnomics} with USD as the base currency\footnote{FX pairs including the USD typically have USD as the base pair ie 1USD = X except for the Australian dollar (AUD) which is quoted as AUD/USD , AUD as the base currency. BIS provides all currencies quotes with USD as the base pair}. The sample period covers January 2000 to February 2021. We begin our sample at this point as it coincides with the growing financialisation of the commodity markets and indeed the the emergence of Emerging Markets (\cite{EMasset}) as an asset class meaning both the commodity futures and some of the chosen currencies exhibit increasing liquidity and market depth from this period.
\subsection{Commodity Data}
The free availability of commodity futures prices is limited and as such we use those provided by \textit{Quandl} (\cite{quandl}). A commodity future is an agreement between parties to trade an amount of a given commodity at a certain point in the future at a given price agreed now. The contract is typically named the month it expires or settles e.g May 2021 contract. This contract only trades for a certain period of time and as such does not offer a price series over the full sample. Quandl provides a continuous timeseries of the contracts where the generic first future contains the price of the contract closest to expiry. The same is provide for the second to expiry and so on. As per \cite{gorton2007}, \cite{famafrench} and others we use the nearby or first futures contract in place of a reliable spot price for the commodity and the second as the far future for $F_1$ and $F_2$ described in the equation above [TODO equation no.]. There are notable issues and omissions in the data. The \textit{Quandl} dataset does not provide expiry dates and for the underlying contracts of the continuous series which are required to annualise the basis in order to make it comparable against other metrics such as returns and interest rates and so needs to be estimated\footnote{We estimate the expiry date by looking at the day in an expiry month where the open interest is at it's lowest, we look at this across all expiry months and take the mode of the expiry day in the month. e.g if open interest drops to it's lowest on the 13th day of the month most often we take this day as the actual expiry date}. Depending on the commodity data for longer maturities are patchy so we are limited to only using the first and second futures. This is not necessarily problematic as these are the most liquid contracts and will contain the required information but it does limit the testing of further contracts as is the case in \cite{mainref} who use contracts out to twelve months which due to liquidity considerations is arguably too long. Furthermore, there are periods where there is missing data for some of the series, we use the most recently available price up to five days or else the period is blacklisted and omitted from the analysis. We excluded palladium completely on this basis even though it is a significant export of South Africa.

\subsection{Adjusted Basis}
We test the theory of storage as discussed in \autoref{sec:theory} and the impact of interest rates and seasonality on the basis using the Fama-French test (\cite{famafrench})
\begin{equation}
\label{eq:ff}
 \frac{F_{t,T} - S_t}{S_t} = \sum_{m=1}^{12} \alpha_m d_m + \beta R_{t,T} + e_{t, T}
\end{equation}
where $d_m$ equals one if the contract expires in a given month. If \autoref{eq:strg2} holds then we would expect a one to one relations ship between interest rates and the basis. This is an important decision on whether to adjust the basis or not. If there is a relationship between the basis and interest rates then this would naturally induce correlation with the exchange rate as the interest rate is inherent in \autoref{eq:uip}. 

