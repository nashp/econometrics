\section{Data}
\label{sec:data}
\subsection{Foreign Exchange \& Country Selection}
We focus on a number of countries and currencies described in the literature as ``Commodity Currencies'' by \cite{cocurrencies} amongst others. These are countries where a significant portion of their terms of trade are made up of commodity exports, particularly raw materials. The United Nations Conference on Trade and Development (UNCTAD) denotes a country as commodity dependent if commodities account for more than $60\%$ of its total merchandise export \cite{codependency}. The impact of the commodity price fluctuations are discussed in \cite{coboomexport} and we select a sample of countries from this paper and \cite{codependency}, based on the liquidity as per the Bank of International Settlements (BIS) \cite{bisturnover} of their currency markets and a floating exchange rate to the US Dollar (USD). In addition to the sample of emerging economies we include the non-emerging economies of Australia, Norway and Canada. We use the BIS USD exchange rate provided by\textit{dbnomics} \cite{dbnomics} with USD as the base currency\footnote{FX pairs including the USD typically have USD as the base pair ie 1USD = X except for the Australian dollar (AUD) which is quoted as AUD/USD , AUD as the base currency. BIS provides all currencies quotes with USD as the base pair}. The sample period covers January 2000 to February 2021. We begin our sample at this point as it coincides with the growing financialisation of the commodity markets (\cite{cofinancialisation}) and indeed the the emergence of Emerging Markets (\cite{EMasset}) as an asset class meaning both the commodity futures and some of the chosen currencies exhibit increasing liquidity and market depth from this period. The full list of countries and currencies is shown in \autoref{tbl:fxiso}.

\subsection{Commodity Data}
The sample of commodities is taken from those that make up a significant portion of a countries exports. We use choose those commodities which are in the top three traded commodities by country as outlined by \cite{codependency}.
The free availability of commodity futures prices is limited and as such we use those provided by \textit{Quandl} (\cite{quandl}).  A commodity future is an agreement between parties to trade an amount of a given commodity at a certain point in the future at a given price agreed now. The contract is typically named the month it expires or settles e.g May 2021 contract. This contract only trades for a certain period of time and as such does not offer a price series over the full sample. Quandl provides a continuous timeseries of the contracts where the generic first future contains the price of the contract closest to expiry. The same is provide for the second to expiry and so on. As per \cite{gorton2007}, \cite{famafrench} and others we use the nearby or first futures contract in place of a reliable spot price for the commodity and the second as the far future for $F_1$ and $F_2$ described in the equation \autoref{eq:strg2}\footnote{Follwing \cite{gorton2007} we define the basis $(F_1/F_2 -1) * 365/(D_2 – D_1)$, where F1 is the nearest futures
contract and F2 is the next nearest futures contract; D1 and D2 are the number of days until the last trading
date of the respective contracts.}. \Cref{tbl:conames} shows the commodities and associated tickers which are prefixed by the exchange on which they trade. There are notable issues and omissions in the data. The \textit{Quandl} dataset does not provide expiry dates and for the underlying contracts of the continuous series which are required to annualise the basis in order to make it comparable against other metrics such as returns and interest rates and so needs to be estimated\footnote{We estimate the expiry date by looking at the day in an expiry month where the open interest is at it's lowest, we look at this across all expiry months and take the mode of the expiry day in the month. e.g if open interest drops to it's lowest on the 13th day of the month most often we take this day as the actual expiry date}. Depending on the commodity data for longer maturities are patchy so we are limited to only using the first and second futures. This is not necessarily problematic as these are the most liquid contracts and will contain the required information but it does limit the testing of further contracts as is the case in \cite{mainref} who use contracts out to twelve months which due to liquidity considerations is arguably too long. Furthermore, there are periods where there is missing data for some of the series, we use the most recently available price up to five days or else the period is blacklisted and omitted from the analysis. We excluded palladium completely on this basis even though it is a significant export of South Africa (\cite{codependency}). For both commodity future prices and foreign exchange rates we use monthly data and end of sample i.e. the last day of the month.

\subsection{Adjusted Basis}
We test the theory of storage as discussed in \autoref{sec:theory} and the impact of interest rates and seasonality on the basis using the Fama-French test (\cite{famafrench})
\begin{equation}
\label{eq:ff}
 \frac{F_{t,T} - S_t}{S_t} = \sum_{m=1}^{12} \alpha_m d_m + \beta R_{t,T} + e_{t, T}
\end{equation}
where $d_m$ equals one if the contract expires in a given month. If \autoref{eq:strg2} holds then we would expect a one to one relations ship between interest rates and the basis. This is an important decision on whether to adjust the basis or not. If there is a relationship between the basis and interest rates then this would naturally induce correlation with the exchange rate as the interest rate is inherent in \autoref{eq:uip}. We present the results in \autoref{sec:ffresult}. Using the US three month treasury bill as the interest rate we note that the interest rate is most significant for gold, with a $\beta$ of close to -1. This model explains a significant amount ($80\%$) of the variance. This tallies with the Fama-French findings \cite{famafrench}\footnote{They suggest a coefficient close to positive 1 but we have a calculated the basis by subtracting the far future from the near future rather than the other way around}. We also note that the interest rate is statistically significant across the agricultural commodities such as corn, wheat, cotton, soy, cocoa and coffee. Across the energies we observe little significance or explanatory power in the interest rate, similarly for other industrial metals such as copper and platinum. \Cref{img:boxplot} offers some clues as to this. We note that those commodities which exhibit a wider distribution, with more outliers are less prone to influence by the interest rate. We suggest that the convenience yield component plays a larger part in these commodities and the interest rate has minimal influence. Using these results we adjust those bases that show significant correlation to the interest rate and to not induce further correlation by adjusting all commodities.



