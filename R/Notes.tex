\documentclass[]{article}
\usepackage{lmodern}
\usepackage{amssymb,amsmath}
\usepackage{ifxetex,ifluatex}
\usepackage{fixltx2e} % provides \textsubscript
\ifnum 0\ifxetex 1\fi\ifluatex 1\fi=0 % if pdftex
  \usepackage[T1]{fontenc}
  \usepackage[utf8]{inputenc}
\else % if luatex or xelatex
  \ifxetex
    \usepackage{mathspec}
  \else
    \usepackage{fontspec}
  \fi
  \defaultfontfeatures{Ligatures=TeX,Scale=MatchLowercase}
\fi
% use upquote if available, for straight quotes in verbatim environments
\IfFileExists{upquote.sty}{\usepackage{upquote}}{}
% use microtype if available
\IfFileExists{microtype.sty}{%
\usepackage[]{microtype}
\UseMicrotypeSet[protrusion]{basicmath} % disable protrusion for tt fonts
}{}
\PassOptionsToPackage{hyphens}{url} % url is loaded by hyperref
\usepackage[unicode=true]{hyperref}
\hypersetup{
            pdftitle={Notes},
            pdfauthor={Peter Nash},
            pdfborder={0 0 0},
            breaklinks=true}
\urlstyle{same}  % don't use monospace font for urls
\usepackage[margin=1in]{geometry}
\usepackage{longtable,booktabs}
% Fix footnotes in tables (requires footnote package)
\IfFileExists{footnote.sty}{\usepackage{footnote}\makesavenoteenv{long table}}{}
\usepackage{graphicx,grffile}
\makeatletter
\def\maxwidth{\ifdim\Gin@nat@width>\linewidth\linewidth\else\Gin@nat@width\fi}
\def\maxheight{\ifdim\Gin@nat@height>\textheight\textheight\else\Gin@nat@height\fi}
\makeatother
% Scale images if necessary, so that they will not overflow the page
% margins by default, and it is still possible to overwrite the defaults
% using explicit options in \includegraphics[width, height, ...]{}
\setkeys{Gin}{width=\maxwidth,height=\maxheight,keepaspectratio}
\IfFileExists{parskip.sty}{%
\usepackage{parskip}
}{% else
\setlength{\parindent}{0pt}
\setlength{\parskip}{6pt plus 2pt minus 1pt}
}
\setlength{\emergencystretch}{3em}  % prevent overfull lines
\providecommand{\tightlist}{%
  \setlength{\itemsep}{0pt}\setlength{\parskip}{0pt}}
\setcounter{secnumdepth}{0}
% Redefines (sub)paragraphs to behave more like sections
\ifx\paragraph\undefined\else
\let\oldparagraph\paragraph
\renewcommand{\paragraph}[1]{\oldparagraph{#1}\mbox{}}
\fi
\ifx\subparagraph\undefined\else
\let\oldsubparagraph\subparagraph
\renewcommand{\subparagraph}[1]{\oldsubparagraph{#1}\mbox{}}
\fi

% set default figure placement to htbp
\makeatletter
\def\fps@figure{htbp}
\makeatother


\title{Notes}
\author{Peter Nash}
\date{18/10/2020}

\begin{document}
\maketitle

\subsection{Notes}\label{notes}

\subsection{Lecture 3 Notes}\label{lecture-3-notes}

For ML we will always need to specify the distribution. We are asking
what value of theta given the observations we have seen from an
assumed/known distribution.

How does the information matrix relate to the Variance? The information
matrix is the second derivative of the the log likelihood, so the slope
of the slope, if we have determined the slope and that in itself is not
steep then it is hard to pinpoint where our solution lies, (thinking
along 2d lines if we have a line of very small slope any small change in
y can lead to a large change in X so hard to pinpoint our variable, if
it is steeper a small change in y will lead to a small change in x so
easier to estimate our paramater) so if the second derivative is small,
or information martix is small, then the variance in our estimate of our
paramater can be large. If the second derivative is large then the
information matrix is large and variance in our estimator is less.

The ML estimator is normally distributed and is efficient.

Log Likelihood: We are assuming independence here in our original notes,
hence the joint distribution is the product of their marginal
distributions. This is not the case for a time-series but we will look
at this later. (Assume a product of their conditional distributions and
marginals)

\subsection{Questions}\label{questions}

The example in Log likelihood
\(L(\theta) = f(y_1, \theta)f(y_2, \theta)f(y_T, \theta)\), we are
assuming each sample is independent, what if this is not the case?
e.g.~if \(y_t\) is not independent of \(y_{t-1}\), e.g.~if we take the
price of an equity this is not indpendent, but if we take the return
this would be independent.

\section{Tutorial 2 Covering Log
Likelihood}\label{tutorial-2-covering-log-likelihood}

We are choosing the value of theta which maximises our probability
density function. Higher joint probability distribution , higher
probability, implies more likely

\url{https://www.youtube.com/watch?v=XepXtl9YKwc\&vl=en\&ab_channel=StatQuestwithJoshStarmer}

towardsdatascience.com/maximum-likelihood-estimation-explained-normal-distribution-6207b322e47f

arg max notation, the argument that maximises
\(arg max_{\theta E \Theta}\) \#\# Lecture 5 Notes

Looking at testing and inference. Could this estimate have happened by
chance. We will look at distributions of the estimators.

If the p-value is small it is unlikely that the estimate was generated
by the true value

p-value think about it as the probability the null hypothesis is true

why is individual null hypthoses a t-test and join and f-test?, how is
this linked to what we saw ,

Joint and individual tests can conflict

trying to link the linear restrictions we impose with what

so our restrictions are our hypotheses. So if \(\beta\) is a kx1 vector
of parameters then \(R\beta = q\) is just like \(\beta_0 = u\),
\(\beta_1 - v\), \(\beta_2 = w\)

QL

\subsection{Live Lecture 19/10/2020 of lecture
4}\label{live-lecture-19102020-of-lecture-4}

Log Likelihood: We are assuming independence here in our original notes,
hence the joint distribution is the product of their marginal
distributions. This is not the case for a time-series but we will look
at this later

Thinking of the null hypothesis, our starting point that this is true,
if this is true and we reject it this is type 1, if this is false and we
accept it then this is type 2, so type 1 is getting our starting point
wrong, type two is accepting when it is false

t-ratio distributed around 0 with a t-distribution if the null
hypothesis is true

We do the exact test only with normal errors and linear restrictions. We
do this if we want to test hypothesis or to test our estimators.

So with the exact test we know the exact distribution rather than using
an asymptotic distribution e.g.~normal. For example a standard normal we
would be looking at += 1.96 where as a t-distribution would be += 2.04,
this means that we would be less likely to reject the null hypothesis.
And remember as samples increase the t-distribution becomes more like
the normal

in \(R \beta = q\) we are doing this purely for the algebra.

Example Slide: \(q = [0, 1]\) is the vector that allows \(\beta_2 =1\)
and \(\beta_3 = -\beta_4\). Multiply out the vectors to see.

Use BIC model selection criteria in general.

Exams will contain output from a statistical programmes and we will be
asked questions on this.

\subsection{Lecture 6 Asymptotic Test
Proceedures.}\label{lecture-6-asymptotic-test-proceedures.}

If we can't get exact results we look at asymptotic approximations, we
are more likely to reject using this

use the F where possible when both F \& \(\chi^2\) are given as the
later is asymptotic and F is exact

\(S(\hat\theta)\), the score vector is the derivative of the LL w.r.t
each of the k elements (number of co-efficents/parameters) of the vector
\(\theta\)

\(\lambda R(\theta)\) is a scalar

Q: why are we doing this restriction?

The restrictions are our hypotheses, if they are trye, the two log
likelihood should be similar \(LL(\theta) - LL(\theta^\star)\), See
lecture 5 above

Q: Maybe speak about how the hypotheses and restrictions are alike

\begin{itemize}
\tightlist
\item
  Test Procedures \(R(\hat\theta)\) has the same structure as
  \(R\hat{\beta} - q\)
\end{itemize}

*Note: The difference between Legrange multiplier is that the variance
covariance matrix in the Wald is evaluated at \(\hat\theta\) versus
\(\theta^{\star}\)

\begin{itemize}
\tightlist
\item
  Source of non linear restrictions
\end{itemize}

Q: What is the restricted and non restricted Model?

Is \(\rho\) the correlation co-efficient?

Where doe our hypothesis
\(H_0: R_m(\theta) = \theta_1\theta_2 - \theta_3 =0\) originate from?

-- Bootstrap

Q: THis can be applied to any of the estimators we've used?

Bootrap food for small sample estimates

ARDL model is what we look at in week 3 of the practicals.

\subsection{Live Lecture 23/10/2020 (lecture
6)}\label{live-lecture-23102020-lecture-6}

Restrictions Hypotheses are the same things,

So tip: Start from a general model and then begin to impose restrictions

e.g Start with normal LRM, but if we observe some autocorrelation then
model this

See Source of non linear restrictions

so see \(y_t = \beta x_t -\rho\beta x_{t-1} + \rho y_{t-1}\)

we can say \(\alpha = \rho\) and \$ \beta\_1 = \alpha  \beta\_0\$

equation 3 is our restricted model, equation 2 is our unrestricted, if
we impose a restriction we reduce the number of variables

Non-linear estimation

Why is it non linear? We are multiplying \(\rho\) and \(\beta\),
multiplicative affect

\begin{longtable}[]{@{}l@{}}
\toprule
\begin{minipage}[t]{0.03\columnwidth}\raggedright\strut
in final equation, generalised least squares.\strut
\end{minipage}\tabularnewline
\begin{minipage}[t]{0.03\columnwidth}\raggedright\strut
\(y_t - \rho y_{t-1}\) is on the left hand side as we know this \(y_t\)
and \(y_{t-1}\) and we make an initial estimate of \rho\strut
\end{minipage}\tabularnewline
\bottomrule
\end{longtable}

The serial correlation of the errors implies that the previous value of
the variable affects the current value

-- Recommended reading

`A guide to Econometrics' KEnnedy Stock and Watson - from Ron Verbeek -
closest to the course

\subsection{Lecture 7 26/10/2020}\label{lecture-7-26102020}

We are looking at what happens when models fail or our assumptions fail

LRM is designed to handle correlation between variables. It will lead to
a higher standard error.

Multicollinearity or issues arising from it are not something to worry
unduly about. They are not one of our main assumptions

Always plot at and look at the residuals, YOU WILL BE PENALISED in your
project if you don't look at outliers, structural breaks etc.

Q: In Gauss LRM: How have you come by the variance of \(\hat \beta\) in
first term

Q: Could you explain more about respecifying the model? \& Dummy
variable trap

Q: Can we test exogeneity simply by looking at the correlation ofX \& u?

Q: With \(\Omega\) we get 0 off diagonals because,
\(E[u_t u_{t-i}] = 0\) \# (serial correlation or autocorrelation)

Q: In GLS we see the distribution contains \(\Omega\), to the following
results are just the result of maximising the Log-Likelihood? Log of
distribution, take the first and second derivatives w.r.t the paramaters
\#\# Live Lecture for lecture 7 26/10/2020

Omitted Variables:

if you've left out a variable/omitted variable or have the wrong
functional form then you have an error in your model.

\subsection{Lecture 8 - Diagnostic
Tests}\label{lecture-8---diagnostic-tests}

Always look at graphs and residuals, these will explain symptoms

serial correlation of errors etc (acf in R)

Question on Tests: Although the justification of these tests is
asymptoic, versions which use the F test seem to work well in practice

Question Structural Stability: In time series will this not vary over
time? What happens then? I think this is addressed, we split the model
into subperiods next

Question: With the restricted model how do we get to \(y = X \beta + u\)
? In the test statistic we revert to \(u_1\) and \(u_2\)

Question: It seems a big assumption to say the variances are the same?
And when we are testing the variance ration and this is not = 1

Question: I don't quite understand the use of dummy variables

Note: Serial correlation: \(\hat\beta\) might be inconsistent if there
are lagged dependent variables.

So what are we doing here? We are trying to assess whether we have the
correct model. Are there unknown breakpoints or structural breaks? Is
there serial correlation in the residuals or there are lagged dependent
variables. Heterskedascticity

We run tests where the null hypothesis is that there is no problem

\subsection{Live Lecture on Lecture 8}\label{live-lecture-on-lecture-8}

Chow test is a test to compare two different estimates subsets of the
same

Question: With the restricted model how do we get to \(y = X \beta + u\)
? In the test statistic we revert to \(u_1\) and \(u_2\):

Because \(\beta_1\) = \(\beta_2\)

we have three models, \$y\_1 \$ \(y_2\) and \(y\). So we have three sets
of residuals

Variance ration: They will never be exactly the same, it is a test of
significance

ok null hypothesis is that the variance is the same, we test if it is
the same within a given level of confidence, say 5\%.

and for example if the variance in our estimate of our parameter for a
certain period is high, then if there was a structural break we would be
more likely to accept this as the same variance because of the

We can also do the above wth dummy variables

This might deal with different variances.

the example we are just binding or combining the matrices.

In the public health Scotland data for Coronavirus.

We have a dummy variable in Care home size. THis is because if this was
a continuous variable there might be large jumps and steps. We bucket
here so it is more consistent and we use a dummy variable. So for a
given observation this will either be 1 or 0. So we will have a variable
for \textless{} 20 , 20-30, 30-40, 50-60 etc.

Note: Unknown breakpoint - searching for a regime change if we do not
know it. These do not have great power however, see \((1-0.95^4)\)

\subsubsection{Week 5 Online Lecture
(2020-11-02)}\label{week-5-online-lecture-2020-11-02}

Stationarity: A distribution can be strongly stationary but not weakly
stationary as they may not have defined moments

Order of integration: The number of times a series must be differenced
before it becomes stationary

Question: Does taking the log make something stationary? Like in our log
gdp life expectancy?

Question: What is AR1? THe lag or univariate?

Question: Variance of \(y_{t}^2\)?

Question: Mean is consant? If this is not stationary then this may not
be true?

Question: Random walks? What are we to take from this? We will need to
check if a series is a random walk?Same with MA, these are particular
cases of univariate models?

\subsubsection{Week 5 Online Lecture}\label{week-5-online-lecture}

Okay so ARIMA (p, d, q) how many times we difference, d, our moving
average q, and our lags p

I don't quite undersdant the detrendint of the data, Frish Waugh
theorem, what is \(y_t\)

\subsubsection{Live Lecure week 5}\label{live-lecure-week-5}

We should be able to write down these processes and estimate them

Common factors: Basically saying that sometimes you can estimate an
ARIMA model but it is actually a random walk, need to check the joint
significance,

If it was a true ARIMA then the coefficients on the AR and MA would be
different and would be jointly significant

Comparing different models via AIC,BIC comes into play here

Frisch-Waugh Slide: Should we include trend in the equation or detrend

Ok so \(y\tilde\) and \(x\tilde\) are the error residuals, so we say we
want to look at what is left over after looking at particular variable,
or controllign for a particular variable what is lef tover, we then
estimate the residuals on this

Trend and difference: Because in the random walk there is no \(\rho\)
which if it is less than one acts as damping factor in the AR model. And
remember if we have AR of order p, then this \(\rho\) will get
multiplied out so like \(\rho^3 y_{t-1}\) so this will become less and
less. Multiply out the expression to see. Also look at previous lecture
notes.

Why don't we want a trend in the change \(\Delta y\)? A trend in change
is quadratic trend in level

This will play a roll with co-integration later on

Unit Roots: Something you need to know if you're doing time series
analysis.

Q: Is there a unit root in log GNP? Do we know? Do we care? paper

Sometimes over the longer period it might be \(I(1)\)

An \(I(0)\) process with a step change will appear \(I(1)\)

Unit Roots:

example \(y_t = \rho y_{t-1} + \epsilon_t\)

For stability the soltuion to \((1 - \rho z) = 0\)

One of the reasons first difference models is good for forecasting , you
do not go back to some equilibrium or trend, might be good for
forecasting but not good for understanding what is going on

Make sure algebra of AR etc, make sure we know restrictions,

The long run multiplier is what the ARDL system looks like after it
settles down, given \(|\rho| < 1\). So in our practical example this was
\(\beta_0 + \beta_1 / (1 - \alpha_1)\). This is known as the long run
elasticity if the variables are in log form. This is a good example
\url{https://www.youtube.com/watch?v=ID96csw4yso\&ab_channel=MarginalRevolutionUniversity}

\end{document}
